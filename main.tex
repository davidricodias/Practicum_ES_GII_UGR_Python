% Prácticas de Estadística para el GII de la UGR, una aproximación con Python
% José David Rico Días
% Bajo Licencia Creative Commons 3.0 (BY-NC-SA)


% CONSIDERACIONES SI NO USAS OVERLEAF   %
% Compilación : XeLaTex                 %
% Lista de paquetes:                    %
%   -   fontspec                        %
%   -   hyperref                        %
%   -                               %
%   -                               %
%   -                               %
%   -                               %

% Configuración de la clase book %
%------------------------------------------------------------------------------%

\documentclass[openany,a4paper]{book}

% Indenta con números hasta 4 %
\setcounter{secnumdepth}{4}
\setcounter{tocdepth}{4}

% Paquetes %
%------------------------------------------------------------------------------%

% Uso de fuentes especiales %
\usepackage{fontspec}
% Configuración de la fuente % Se usa la fuente San Francisco; la que usa Apple %
\setmainfont[
Path=./fonts/,      % Directorio de las fuentes %
BoldFont=SF-Bold.otf,
ItalicFont=SF-Italic.otf,
BoldItalicFont=SF-Bold-Italic.otf
]{SF.otf}

% Permite hipervínculos en el PDF, lo que facilita la navegación %
\usepackage[breaklinks=true,hidelinks]{hyperref}

% Gráficos % Insertar imágenes %
\usepackage{graphicx}

% Ajuste de márgenes. Optimizado para lectura en ordenador %
\usepackage[papersize={210mm,297mm},lmargin=2cm,rmargin=2cm,top=2cm,bottom=2cm,headsep=3mm]{geometry}
% Con esta configuración conseguimos que el PDF se pueda leer en un e-book. %
% Puede que no sea una feature muy interesante.                             %
%
%       \usepackage[papersize={95mm,125mm},lmargin=1.5mm,rmargin=1.5mm,top=7mm,bottom=1.5mm,headsep=3mm]{geometry}

% Idioma %
\usepackage[spanish]{babel}

% Introducir código de python %
\usepackage{listings}       %PARA EJEMPLOS DE CÓDIGO%
\usepackage{xcolor}
 
\definecolor{codegreen}{rgb}{0,0.6,0}
\definecolor{codegray}{rgb}{0.5,0.5,0.5}
\definecolor{codepurple}{rgb}{0.58,0,0.82}

\lstdefinestyle{mystyle}{
    backgroundcolor=\color{white},   
    commentstyle=\color{codegreen},
    keywordstyle=\color{magenta},
    numberstyle=\tiny\color{codegray},
    stringstyle=\color{codepurple},
    basicstyle=\ttfamily\footnotesize,
    breakatwhitespace=false,         
    breaklines=true,                 
    captionpos=b,                    
    keepspaces=true,                 
    numbers=left,                    
    numbersep=3pt,                  
    showspaces=false,                
    showstringspaces=false,
    showtabs=false,                  
    tabsize=1
}

\lstset{style=mystyle}
% Comandos propios %
%------------------------------------------------------------------------------%

% Script para la creación de citas a un personaje %
\newcommand{\chapquote}[3]{\begin{quotation} \textit{#1} \end{quotation} \begin{flushright} - #2, \textit{#3}\end{flushright} } 

%------------------------------------------------------------------------------%

\begin{document}
 
\begin{titlepage}
	\begin{center}		
		\vspace*{1.0cm}
		\begin{Huge}
			\textbf{Prácticas de estadística para el GII UGR, una aproximación con Python}\\
		\end{Huge}	
		\vspace*{1.0cm}
		\rule{150mm}{0.1mm}\\
		
		\vspace*{1.5cm}
		\begin{figure}[htb]
			\begin{center}
				\includegraphics[width=9cm]{images/logotypes/Logo UGR.png}
			\end{center}
		\end{figure}
        \huge{José David Rico Días}\\
		\vspace*{1cm}
	    \Large{\today}
	\end{center}		
\end{titlepage}

\thispagestyle{empty}

Este documento está bajo la licencia CC BY-NC-SA. Junto a este documento debes haber recibido una copia de la licencia correspondiente. Si no es así puedes consultar la licencia en el siguiente enlace: \textbf{\href{https://creativecommons.org/licenses/by-nc-sa/4.0/legalcode.es}{https://creativecommons.org/licenses/by-nc-sa/4.0/legalcode.es}}

This document is under the CC BY-NC-SA license. Among these document you may have received a license's copy. If not, you can read the license in the following link: \textbf{\href{https://creativecommons.org/licenses/by-nc-sa/4.0/legalcode.en}{https://creativecommons.org/licenses/by-nc-sa/4.0/legalcode.en}}

\newpage

\thispagestyle{empty}

\vspace*{9cm}
\chapquote{"Python is an experiment in how much freedom programmers need. Too much freedom and nobody can read another's code; too little an expressiveness is endangered."}{Guido von Rossum}{Creador de Python}


\chapter*{Motivación}

\thispagestyle{empty}

Durante el inicio del del 2º cuatrimestre, empecé las prácticas de Estadística. Comenzamos aprendiendo el lenguaje de programación R, concretamente R Commander. Yo que ya conocía Python y las famosas bibliotecas SciPy y NumPy, pensé que con la versatilidad que ofrece este maravilloso lenguaje, sería más adecuado para un Ingeniero Informático.

Y así lo hice.

\chapter*{Sobre este manual}

\thispagestyle{empty}

Es recomendable que trabajes con este manual en PDF en el ordenador. Hay links en el texto que permiten ir directamente a las páginas. 

Además, puedes consultar el GitHub con todos los datos y las soluciones a los ejercicios.

\tableofcontents

\newpage


\chapter{Introducción}

\setcounter{page}{1}

\section{¿Qué es Python?}

\section{Nuestro entorno de trabajo}

Para las prácticas usaremos un IDE online: \href{https:\\ide.cs50.io}{ide.cs50.io}. Este IDE fue creado por la Universidad de Harvard, permitiendo programar en Python y en C.
Además proporciona un \textit{sandbox} Linux (Ubuntu concretamente), todas las librerías necesarias y una API para crear tests sobre tus programas.

Debemos registrarnos con una cuenta de \href{https://github.com/}{GitHub}.





\chapter{Manos a la obra}

\end{document}